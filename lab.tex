\section{Introducción}
Este es un reporte de trabajo donde se discute la resolución de un problema propuesto como ejemplo en el libro \quotes{Analisis cuantitativo con WinQSB}\cite{acwinqsb}.

El método de solución se divide en tres. Por un primer lado, se dará solución al problema propuesto por los autores sin alteraciones. Luego, se modificará ligeramente los datos del mismo para una segunda resolución. Por último, utilizando el software WinQSB, se dará una última solución al problema original, acompañado de un análisis e interpretación de estos resultados.

\subsection{Problema a desarrollar}
\insertimage[\label{img:problema}]{assets/Problema.png}{scale=0.6}{Problema a desarrollar.}

El problema en cuestión corresponde al ejemplo 7-1 del libro, que corresponde a un modelo de inventario simple para el cual no existe agotamiento y el tiempo de reposición es inmediato.

%%%%% Fin seccion introduccion %%%%%
\clearpage

\section{Marco teórico}
Se dispone a continuación de un breve marco teórico con el que se verá la resolución del problema.
\subsection{Modelos de Inventario}
Según el apunte número 2 encontrado en la plataforma Reko\cite{apunte}, un modelo de inventario es un modelo matemático que describe el comportamiento de un sistema de inventarios. Estos modelos se pueden ocupar para derivar políticas óptimas de inventario, las que suelen ser implementadas a partir de sistemas computacionales que indican cuando conviene reabastecer.

Estos modelos pueden llegar a ser áltamente refinados y de múltiples variables, acontando por cantidad de agotamiento de inventario permitida y el costo que este incorre, demandas no constantes, tiempos de reabastecimientos no inmediatos, entre otros.

\subsection{Lote de Wilson}
\insertimage[\label{img:wilson}]{assets/Wilson.png}{scale=0.6}{Lote de Wilson simple, elaboración propia.}
El lote de Wilson corresponde al Lote económico de compra óptimo para un modelo de inventario simple sin agotamiento y reabastecimiento inmediato, además de demanda constante, tal y como se presenta en la figura (\ref{img:wilson}).

En esta figura, se presenta una figura triangular con una hipotenusa $\lambda$. Esta hipotenusa corresponde a la demanda del producto inventariado. Por otro lado, el valor de $Q$ corresponde a nuestro Lote económico de compra óptimo en un periodo $t$ dado. La hipotenusa en un instante $t$ en concreto corresponde a la cantidad de inventario en ese momento.

\subsection{Funcion de Costo en un periodo}.
\insertimageleft[\label{img:grafica-ft}]{assets/graf-ft.png}{0.4}{Grafica de $f_t(Q)$ para los datos del problema dado (Figura \ref{img:problema}). En el eje de las abcisas, el valor de Q. En el eje de las ordenadas, el costo. Elaborado en el software Geogebra.}

Esta función de costo corresponde a todos los costos que se incurren en un periodo $t$ determinado. Para efectos del modelo de Wilson, corresponde a la suma de tres costos en concreto:

\begin{enumerate}
    \item Costo de adquisición, o la cantidad de producto multiplicado por su costo individual
    \item Costo de mantención de inventario, calculable como el costo de mantención de una unidad de inventario multiplicada por la cantidad promedio de inventario en el periodo
    \item Costo de colocar un pedido nuevo, que puede darse por costos bureocráticos y otros costos del negocio.
\end{enumerate}

A partir de estos tres costos podemos describir la Funcion de costo en un periodo de la siguiente manera:\\ \\

\insertgathered[\label{eqn:ft}]{
    f_t(Q) = \textrm{Costo de Adquisición} + \textrm{Costo de Inventario} + \textrm{Costo de Pedido}\\
    f_t(Q) = C_a Q + \frac{1}{2} Q t C_i + C_p
}


En esta ecuación, los valores de $C_a$ corresponden al costo de adquisición de una unidad de producto, $C_i$ es el costo de mantención de una unidad de producto, y $C_p$ es el costo de hacer un pedido. Notar además que los valores que acompañan a $C_i$ corresponden al area de la figura que el modelo de inventario intenta describir. La que es presentada en la equación \refeq{eqn:ft} corresponde al area de la figura para el modelo de Wilson.

Además, Q corresponde a una cantidad de inventario que se compra en un periodo. Nuestro objetivo es optimizar este valor Q, pero nuestra función de costo no es convexa, por lo que debemos encontrar una función que si lo sea.

\clearpage

\subsection{Función de Costo por unidad de Tiempo}
\insertimageleft[\label{img:grafica-fut}]{assets/graf-fut.png}{0.4}{Grafica de $f_{ut}(Q)$ para los datos del problema dado (Figura \ref{img:problema}). En el eje de las abcisas, el valor de Q. En el eje de las ordenadas, el costo. Elaborado en el software Geogebra.}

A partir de nuestra función de costo por periodo \ref{eqn:ft}, podemos generar una nueva función de costo tal que:

\insertgathered[\label{eqn:futuno}]{
    f_{ut}(Q) = \frac{f_t(Q)}{t} \\
    f_{ut}(Q) = \frac{C_a Q}{t} + \frac{\frac{1}{2} Q t C_i}{t}  + \frac{C_p}{t}
}

Jugando un poco con la figura de nuestro modelo de Wilson \ref{img:wilson}, podemos obtener un valor de $t$ que sea más útil para desarrollar nuestra ecuación.

\insertgathered[\label{eqn:tan}]{
    \tan \alpha = \lambda = \frac{Q}{t} \\
    t = \frac{Q}{\lambda}
}

Reemplazando este nuevo valor de $t$, podemos desarrollar nuestra ecuación inicial \ref{eqn:futuno}, tal que:

\insertgathered[\label{eqn:fut}]{
    f_{ut}(Q) = \frac{C_a Q}{\frac{Q}{\lambda}} + \frac{\frac{1}{2} Q t C_i}{t}  + \frac{C_p}{\frac{Q}{\lambda}} \\
    f_{ut}(Q) = \frac{C_a Q \lambda}{Q} + \frac{\frac{1}{2} Q t C_i}{t}  + \frac{C_p \lambda}{Q} \\
    f_{ut}(Q) = C_a \lambda + \frac{1}{2} Q C_i  + \frac{C_p \lambda}{Q}
}

Esta función es convexa, por lo que podemos igualar el valor de la derivada de esta a cero para obtener una expresión que nos entrega un Lote económico de compra óptimo para esta figura.

\insertgathered[\label{eqn:dfut}]{
    \frac{d f_{ut}(Q)}{d Q} = 0 \\
    \frac{d (C_a \lambda)}{d Q} + \frac{d (\frac{1}{2} Q C_i)}{d Q} + \frac{d (\frac{C_p \lambda}{Q})}{d Q}  = 0 \\
    \frac{1}{2} C_i - \frac{C_p \lambda}{Q²} = 0 \\
    Q² = \frac{C_p \lambda}{\frac{1}{2} C_i} \\
}

\insertequation[\label{eqn:lotewilson}]{\boxed{Q = \sqrt{\frac{2 C_p \lambda}{C_i}}}}

La ecuación resultante corresponde al Lote de Wilson.

%%%%% Fin seccion Marco teórico %%%%%
%\clearpage

\section{Resolución del problema}
\subsection{Problema con los datos entregados}
El problema original (Figura \ref{img:problema}) se plantea un modelo de Wilson simple con los siguientes parámetros:

\begin{itemize}
    \item ($C_a$) Costo de adquisición: \$20 por unidad.
    \item ($C_p$) Costo de pedido: \$5 por pedido.
    \item ($C_i$) Costo de mantención de inventario: \$4 por unidad.
    \item ($\lambda$) Demanda del producto: 1000 unidades.
\end{itemize}

Utilizando nuestra ecuación para el Lote de Wilson, podemos encontrar nuestro lote económico de compra reemplazando los parámetros correspondientes:

\insertequationanum{Q = \sqrt{\frac{2 \times 5 \times 1000}{4}}}
\insertequation[\label{res:lec1}]{\boxed{Q = 50}}

Sabemos entonces que optimizamos los costos de nuestros pedidos con una cantidad de 50 unidades. Con este resultado, podemos verificar cual es el costo por periodo (Utilizando la ecuación \ref{eqn:ft}):

\insertequationanum{f_t(50) = 20 \times 50 + \frac{1}{2} \times 50 \times 4 \times t + 5}

Podemos reemplazar el valor de $t$ nuevamente (utilizando la ecuación \ref{eqn:tan}), con lo que nos quedariamos con:

\insertequationanum{f_t(50) = 20 \times 50 + \frac{1}{2} \times 50 \times 4 \times \frac{50}{1000} + 5}
\insertequation[\label{res:ft1}]{\boxed{f_t(50) = \$1010}}

Esta comprobación también es aplicable a nuestra función de costo por unidad de tiempo (Equación \ref{eqn:fut}):

\insertequationanum{f_{ut}(50) = 20 \times 1000 + \frac{50 \times 4}{2}   + \frac{5 \times 1000}{50}}
\insertequation[\label{res:fut1}]{\boxed{f_{ut}(50) = \$20200}}

Finalmente, con esta cantidad de pedidos nos podemos asegurar de que en nuestra unidad de tiempo conseguimos el menor costo posible. Esta unidad de tiempo puede representar un año, dividido en periodos $t$. Con una simple división entre el resultado de nuestra ecuación \ref{res:fut1} y nuestra ecuación \ref{res:ft1}, notamos que la cantidad de periodos es de 20, lo que significa que en esta unidad de tiempo de un año haríamos un total de 20 pedidos a intervalos constantes, lo que equivale a un pedido cada 18.25 días, aproximadamente.

\subsection{Problema con datos originales}
Vamos a generar otra resolución del problema con datos originales, según las indicaciones del laboratorio. Estos datos se presentan a continuación:

\begin{itemize}
    \item ($C_a$) Costo de adquisición: \$35 por unidad (Esto es dado por la suma de los dígitos de mi RUT).
    \item ($C_p$) Costo de pedido: \$7 por pedido.
    \item ($C_i$) Costo de mantención de inventario: \$3 por unidad.
    \item ($\lambda$) Demanda del producto: 1200 unidades.
\end{itemize}

Estos cambios de parámetros no cambiar el modelo mayormente, por lo que el reemplazo que se hizo en la sección anterior es igual de válido. Bajo esto, tenemos entonces un nuevo valor para nuestro lote económico de compra:

\insertequationanum{Q = \sqrt{\frac{2 \times 7 \times 1200}{3}}}
\insertequation[\label{res:lec2}]{\boxed{Q = 74.83 \approx 75}}

Debido a los valores decimales, utilizaremos la aproximación de $Q = 75$ para las funciones, presumiendo que no se pueden adquirir unidades decimales del producto. Notar que esto no sería el caso si nuestra unidad de adquisición fueran litros o gramos de producto, donde tiene sentido el uso de decimales.

A partir de nuestro Lote económico de compra adquirido, podemos calcular el costo con nuestra nueva función de costo por periodo de tiempo, tal que:

\insertequationanum{f_t(75) = 35 \times 75 + \frac{1}{2} \times 75 \times 3 \times \frac{75}{1200} + 7}
\insertequation[\label{res:ft2}]{\boxed{f_t(75) = \$2639.03125 \approx \$2639}}

Por último, nuestro costo por unidad de tiempo sería:

\insertequationanum{f_{ut}(75) = 35 \times 1200 + \frac{75 \times 3}{2}   + \frac{7 \times 1200}{75}}
\insertequation[\label{res:fut2}]{\boxed{f_{ut}(75) = \$42224.5 \approx \$42225}}

Notar que esta vez, la división entre nuestro costo por unidad de tiempo y nuestro costo por periodo para la obtención de periodos en nuestra unidad de tiempo nos entrega como resultado 16\footnote{Esto, utilizando el valor real (con decimales) de cada función. El valor con la aproximación es de 16.000378931}. Para términos de duración de estos periodos, sería de 22.8125, lo que equivale a 1 pedido cada, aproximadamente, 23 días.

\subsection{Problema con datos entregados, via WinQSB}
En nuestra última resolución, utilizaremos el problema original (Figura \ref{img:problema}) con datos sin alterar. La resolución esta vez se hará por vías del software WinQSB, montado en una máquina virtual de Windows XP. Utilizando el programa de problemas de inventario, generaremos una nueva solución de acuerdo a la siguiente figura:

\insertimage[\label{img:winqsb:1}]{assets/winqsb/1.png}{scale=0.6}{Selección de tipo de problema de inventario.}

Ingresamos los datos del problema original de la siguiente forma, y generamos la solución:
\insertimage[\label{img:winqsb:2}]{assets/winqsb/2.png}{scale=0.6}{Tabla de datos de entrada para el problema de inventario.}

Nuestra tabla de resultado muestra, en su primera mitad, la tabla de datos originales, y luego las soluciones correspondientes. Estas soluciones son iguales a los datos encontrados anteriormente, o derivaciones del mismo:
\insertimage[\label{img:winqsb:3}]{assets/winqsb/3.png}{scale=0.6}{Tabla de resultados generada en WinQSB}

Siguiendo desde arriba hacia abajo en nuestra tabla presentada en la figura \ref{img:winqsb:3}, vemos:

\begin{enumerate}
    \item \textbf{Order quantity:} Correspondiente al lote económico de compra que hemos calculado.
    \item \textbf{Maximum inventory:} Igual a nuestra cantidad de orden, debido a la naturaleza simple de nuestro modelo.
    \item \textbf{Maximum backorder:} No aplica para este problema.
    \item \textbf{Order interval in year:} Corresponde a 1 sobre la cantidad de periodos. Nuestra solución original sugiere 20 periodos de compra, por lo que el intervalo de compra en esta unidad de tiempo es de $0.05$.
    \item \textbf{Reorder point:} No aplica para este problema
    \item \textbf{Total setup or ordering cost:} Costo total en realización de pedidos. Hacemos un total de 20 pedidos a un costo de 5 cada uno, por lo que nuestro costo es de \$100 unidades monetarias.
    \item \textbf{Total holding cost:} Costo total de mantener pedidos. Puede ser calculado como el area de la figura (El area de un triangulo, según la figura \ref{img:wilson}) por el costo de mantener el inventario, dividido por el periodo. El resultado es de \$100 unidades monetarias.
    \item \textbf{Total shortage cost:} No aplica para este problema.
    \item \textbf{Subtotal of above:} Suma de nuestro costo de pedidos mas nuestro costo de mantención de inventario en nuestra unidad de tiempo.
    \item \textbf{Total material cost:} Cantidad de pedidos, por cantidad de orden, por costo de adquisición. Nuestro resultado corresponde a \$20,000 unidades monetarias.
    \item \textbf{Grand total cost:} Costo total por unidad de tiempo. Según nuestro resultado inicial (Ecuación \ref{res:fut1}), \$20,200 unidades monetarias.
\end{enumerate}

Podemos complementar además esto, con un análisis gráfico (Figura \ref{img:winqsb:4}). A continuación se presenta un gráfico de cada tipo de costo dependiendo de la cantidad de orden. En el eje de las abcisas, la cantidad de orden. En el eje de las ordenadas, el costo en unidades monetarias. En rosa, costo de pedidos. En rojo, costo de mantención de inventario. En negro, costo total por unidad de tiempo. El valor de nuestro lote economico de compra es demarcado como \quotes{EOQ} por el software.

\insertimage[\label{img:winqsb:4}]{assets/winqsb/4.png}{scale=0.38}{Funciones de costo a partir de cantidad de orden}

Como último gráfico, podemos re hacer nuestra figura del modelo de Wilson (Figura \ref{img:wilson}) con los datos obtenidos en nuestro resultado. Nuestro modelo de inventario, entonces, luce de la siguiente manera:

\insertimage[\label{img:winqsb:5}]{assets/winqsb/5.png}{scale=0.38}{Modelo de Wilson a partir de resultados}


%%%%% Fin seccion Resolución del problema %%%%%
\clearpage

\section{Conclusión}

%%%%% Fin seccion Conclusión %%%%%
\clearpage


\bibliography{library}
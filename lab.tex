\section{Introducción}
La tarea de simular un sistema consiste en, a traves del uso del azar y la probabilidad, utilizar un modelo del sistema para la predicción del comportamiento de este mismo. Esto permite un análisis profundo de un sistema sin tener que producir su comportamiento, sino unicamente reproducirlo en un ambiente controlado, con lo que podemos predecir desempeño de este sistema, casos críticos, y otros factores de interés para el mismo.

En el caso a mano, se nos presentan dos tipos de sistemas similares, basados en colas de atención para clientes/usuarios. A través de una simulación de estos, podemos predecir el comportamiento de la atención de ambos sistemas, y sacar conclusiones del desempeño del mismo, e incluso ayudar en la toma de decisiones respecto a este sistema.

Para ambos problemas, se utilizará WinQSB para la simulación de estos.

\subsection{Problema 1}
Este problema es encontrado en el libro Análisis cuantitativo con WinQSB \cite{acwinqsb}. Podemos encontrar un enunciado en el capítulo 13 del mismo.

\insertimage[\label{problema:1}]{assets/problema.png}{scale=0.5}{Enunciado del problema encontrado en el capítulo 13 del libro Análisis cuantitativo con WinQSB \cite{acwinqsb}}

\subsection{Problema 2}
El segundo problema se encuentra en los requerimientos para este informe. A continuación se presenta con una transcripción del mismo: \\

\noindent \fbox{
    \parbox{\textwidth}{
        \label{problema:2}
        Una sala de cine posee dos cajas, que atienden a un cliente en
        un promedio de 17 minutos con una desviación de 0.012. Los
        clientes llegan a una tasa de uno cada 14 minutos y hacen una
        sola cola con capacidad máxima de 16 clientes y disciplina
        FIFO.

        Considere que la llegada de los clientes se comporta similar a
        una distribución tipo Poisson y los cajeros con una distribución
        normal.

        Se pide:

        \begin{itemize}
            \item Simular 150 minutos con una semilla Random igual a los primeros
                  5 dígitos del Rut del alumno (Por ejemplo, Rut 22.167.744-7,
                  semilla es 22167)
            \item Mostrar los resultados e interpretación de cada uno de los
                  resultados que se muestran en los cuadros: \begin{itemize}
                      \item Show Customer Analysis
                      \item Show Server Analysis
                      \item Show Queue Analysis
                  \end{itemize}
        \end{itemize}
    }
}

%%%% FIN SECCIÓN INTRODUCCIÓN %%%%
\clearpage

% \section{Marco teórico}
\section{Simulación del problema}
Ambos problemas son similares en el sentido de que son problemas de simulación de colas de atención, por lo que el software de WinQSB que se utilizará será, precisamente, Queuing System Simulator (Simulador de sistemas de colas).
\subsection{Simulación del problema uno}
Según la especificaciones del problema, podemos detectar un total de 4 componentes: El cajero 1, El cajero 2, Los clientes y la Cola. Por lo mismo, nuestra generación del problema dentro del software verá reflejado esto:

\insertimage[\label{p1:1}]{assets/Problema 1/1.png}{scale=0.75}{Generación de una nueva simulación de un sistema de colas.}

En este software existen 4 tipos de componentes. \quotes{C} corresponde a \quotes{Customer arriving source} (Fuente de llegada de clientes), \quotes{S} corresponde a los servidores, \quotes{Q} corresponde a las colas. Por último, \quotes{G}, corresponde a recolectores de basura, que no aplica para este problema. Con esta información, podemos clasificar a los componentes de la siguiente manera:
\insertimage[\label{p1:2}]{assets/Problema 1/2.png}{scale=0.75}{Descripción de cada tipo de componente}

Podemos rellenar la tabla de descripción del comportamiento del sistema a partir de los datos entregados en el mismo (\ref{problema:1}). Se han omitido algunas de las columnas de esta tabla que no son relevantes para el problema.
\insertimage[\label{p1:3}]{assets/Problema 1/3.png}{scale=0.6}{Tabla de descripción del comportamiento del sistema}

Podemos generar, entonces, la simulación con estos datos. Para este caso en concretos, utilizaremos la semilla por defecto para simular un total de 100 minutos:
\insertimage[\label{p1:4}]{assets/Problema 1/4.png}{scale=0.75}{Menú de generación de simulación}

\clearpage
A continuación, en las figuras \ref{p1:5}, \ref{p1:6} y \ref{p1:7}, se muestras las tablas resultantes de nuestra simulación\footnote{Debido a la configuración de mi máquina virtual, particularmente en uno de los archivos .dll que maneja números flotantes en sistemas antiguos, que tuvo que ser reemplazado por uno descargado de internet para el funcionamiento del software, no logro replicar números decimales exactamente en esta máquina con respecto a otras. Por favor, tengan presente un ligero error no determinado en los valores decimales de mis resultados.}. En primer lugar, tenemos los resultados desde la perspectiva de los clientes, lo que nos muestra tiempos de espera, largo máximo de la cola, entre otros valores. En segundo, vemos el porcentaje de uso de los servidores a lo largo de la simulación. Por último, se ve un poco más en detalle las descripciones de la cola misma.
\insertimage[\label{p1:5}]{assets/Problema 1/5.png}{scale=0.75}{Análisis de clientes}
\insertimage[\label{p1:6}]{assets/Problema 1/6.png}{scale=0.7}{Análisis de servidores}
\insertimage[\label{p1:7}]{assets/Problema 1/7.png}{scale=0.75}{Análisis de cola}

En total, podemos ver que la simulación recibió a un total de 1123 clientes, de los cuales 871 fueron atendidos, lo que equivale a una taza de atención del $0.76$, o al rededor de 3 cada 4 clientes que llegaron. La espera promedio fue de al rededor de 11 minutos, con una atención promedio de 4 minutos por cliente, por cajero. Esto se traduce a un uso promedio del 29.03\% del tiempo de cada cajero en atención de clientes. Por último, vemos que el largo máximo de la cola fue alcanzado, con un valor de 15 clientes en ella en su peak, sin embargo, el largo promedio de esta fue de tan solo 1.65 clientes. La espera más alta que se generó en esta simulación fue de casi 35 minutos.

\clearpage
\subsection{Simulación del problema dos}

\section{Conclusión}

\bibliography{library}
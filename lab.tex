\section{Introducción}
Este es un reporte de trabajo donde se discute la resolución de un problema propuesto como ejemplo en el libro \quotes{Analisis cuantitativo con WinQSB}\cite{acwinqsb}.

El método de solución se divide en tres. Por un primer lado, se dará solución al problema propuesto por los autores sin alteraciones. Luego, se modificará ligeramente los datos del mismo para una segunda resolución. Por último, utilizando el software WinQSB, se dará una última solución al problema original, acompañado de un análisis e interpretación de estos resultados.

\subsection{Problema a desarrollar}
\insertimage[\label{img:problema}]{assets/Problema.png}{scale=0.6}{Problema a desarrollar.}

El problema en cuestión corresponde al ejemplo 7-1 del libro, que corresponde a un modelo de inventario simple para el cual no existe agotamiento y el tiempo de reposición es inmediato.

\clearpage


\section{Marco teórico}
Se dispone a continuación de un breve marco teórico con el que se verá la resolución del problema.
\subsection{Modelos de Inventario}
\lipsum[1]

\subsection{Lote de Wilson}
\insertimage[\label{img:wilson}]{assets/Wilson.png}{scale=0.6}{Lote de Wilson simple, elaboración propia.}
\lipsum[2]

\clearpage
\bibliography{library}